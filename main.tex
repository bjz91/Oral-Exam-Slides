% Copyright 2004 by Till Tantau <tantau@users.sourceforge.net>.
%
% In principle, this file can be redistributed and/or modified under
% the terms of the GNU Public License, version 2.
%
% However, this file is supposed to be a template to be modified
% for your own needs. For this reason, if you use this file as a
% template and not specifically distribute it as part of a another
% package/program, I grant the extra permission to freely copy and
% modify this file as you see fit and even to delete this copyright
% notice. 

% \documentclass{beamer}
\documentclass[handout]{beamer} % Disable animation temporally

% Replace the \documentclass declaration above
% with the following two lines to typeset your 
% lecture notes as a handout:
%\documentclass{article}
%\usepackage{beamerarticle}


% There are many different themes available for Beamer. A comprehensive
% list with examples is given here:
% http://deic.uab.es/~iblanes/beamer_gallery/index_by_theme.html
% You can uncomment the themes below if you would like to use a different
% one:
%\usetheme{AnnArbor}
%\usetheme{Antibes}
%\usetheme{Bergen}
%\usetheme{Berkeley}
%\usetheme{Berlin}
%\usetheme{Boadilla}
%\usetheme{boxes}
%\usetheme{CambridgeUS}
%\usetheme{Copenhagen}
%\usetheme{Darmstadt}
%\usetheme{default}
%\usetheme{Frankfurt}
%\usetheme{Goettingen}
%\usetheme{Hannover}
%\usetheme{Ilmenau}
%\usetheme{JuanLesPins}
%\usetheme{Luebeck}
%\usetheme{Madrid}
\usetheme{Malmoe}
%\usetheme{Marburg}
%\usetheme{Montpellier}
%\usetheme{PaloAlto}
%\usetheme{Pittsburgh}
%\usetheme{Rochester}
%\usetheme{Singapore}
%\usetheme{Szeged}
%\usetheme{Warsaw}

\usepackage{appendixnumberbeamer} % New beamer numbers in appendix

\title[]{\large Estimation of improved PM$_{2.5}$ exposures from satellite and low-cost sensors for use in a population-based study of PM$_{2.5}$ and renal disease}

% A subtitle is optional and this may be deleted
%\subtitle{Optional Subtitle}

\author[]{Jianzhao Bi\inst{1} \and \textcolor[rgb]{0.1,0.6,0.1}{Yang Liu}\inst{1}\inst{*}}
% - Give the names in the same order as the appear in the paper.
% - Use the \inst{?} command only if the authors have different
%   affiliation.

\institute[Emory University] % (optional, but mostly needed)
{
  \inst{1}%
  Environmental Health Sciences\\
  Emory University
 }
% - Use the \inst command only if there are several affiliations.
% - Keep it simple, no one is interested in your street address.

\date{\today}
% - Either use conference name or its abbreviation.
% - Not really informative to the audience, more for people (including
%   yourself) who are reading the slides online

%\subject{Theoretical Computer Science}
% This is only inserted into the PDF information catalog. Can be left
% out. 

% If you have a file called "university-logo-filename.xxx", where xxx
% is a graphic format that can be processed by latex or pdflatex,
% resp., then you can add a logo as follows:

\pgfdeclareimage[height=1cm]{logo}{img/logo}
\logo{\pgfuseimage{logo}}

% Remove navigation symbols
\setbeamertemplate{navigation symbols}{}

% Add page numbers
\addtobeamertemplate{navigation symbols}{}{%
    \usebeamerfont{footline}%
    \usebeamercolor[fg]{footline}%
    \hspace{1em}%
    \insertframenumber/\inserttotalframenumber
}


% Let's get started
\begin{document}

\begin{frame}
  \titlepage
\end{frame}

% Section and subsections will appear in the presentation overview
% and table of contents.
\section{Background and Objectives}
\subsection{Background}
\begin{frame}{Background}
    \begin{itemize}
        \item<1-> PM$_{2.5}$ is known to trigger or worsen various health outcomes (\textit{e.g.,} heart attack, asthma, and other respiratory problems)
        \item<2-> Regulatory air quality stations for PM$_{2.5}$ measurements lack spatiotemporal coverage
        \item<3-> Satellite \textbf{aerosol optical depth (AOD)} has been increasingly applied to supplement ground PM$_{2.5}$ measurements
    \end{itemize}
    \vspace{-0.25cm}
    \begin{figure}
        \centering
        \includegraphics[width=0.83\textwidth]{img/satellite_aod.jpg}
        \label{fig:pm25}
    \end{figure}
\end{frame}

\begin{frame}{Problems}
    \setbeamercovered{transparent} 
    \begin{enumerate}
        \item<1> Non-random missing satellite AOD caused by cloud cover and high surface brightness (\textit{e.g.,} snow/ice) 
            \begin{itemize}
                \item \textcolor[rgb]{1,0.4,0}{Cloud/snow interacting with AOD levels}
                \item \textcolor[rgb]{1,0.4,0}{No gap-filling approach considering snow-AOD interaction}
            \end{itemize}
        \textcolor[rgb]{0.1,0.1,0.6}{Cloud and snow led to a $\sim$80\% missing AOD data in New York State in 2015}
    \end{enumerate}
    \vspace{-0.1cm}
    \begin{figure}
        \centering
        \includegraphics[width=0.6\textwidth]{img/missing.jpg}
    \end{figure}
\end{frame}

\begin{frame}{Problems}
    \setbeamercovered{transparent} 
    \begin{enumerate}
        \setcounter{enumi}{1}
        \item Sparse regulatory monitoring networks which limit PM$_{2.5}$ estimations with spatial details
            \begin{itemize}
                \item \textcolor[rgb]{1,0.4,0}{Inaccurate estimations in the areas without ground stations}
                \item \textcolor[rgb]{1,0.4,0}{Underuse of low-cost, portable PM$_{2.5}$ sensors (\textit{e.g.,} PurpleAir)}
            \end{itemize}
        \textcolor[rgb]{0.1,0.1,0.6}{In California, there are 629 PurpleAir and 154 EPA stations}
    \end{enumerate}
    \vspace{-0.2cm}
    \begin{figure}
        \centering
        \includegraphics[width=0.48\textwidth]{img/bayarea.jpg}
    \end{figure}
\end{frame}

\begin{frame}{Problems}
    \setbeamercovered{transparent} 
    \begin{enumerate}
        \setcounter{enumi}{2}
        \item Evidence showing the relationship between long-term PM$_{2.5}$ exposure and renal function decline
            \begin{itemize}
                \item \textcolor[rgb]{1,0.4,0}{Unknown effect of short-term PM$_{2.5}$ exposure on the kidney}
                \item \textcolor[rgb]{1,0.4,0}{Worthy of being studied with improved PM$_{2.5}$ exposures}
            \end{itemize}
        \textcolor[rgb]{0.1,0.1,0.6}{Inhaled inert gold nanoparticles enter the bloodstreams and are detected in the urine within minutes after exposure \textit{\small (Miller et al., ACS Nano, 2017)}}
        \begin{figure}
            \centering
            \includegraphics[width=0.8\textwidth]{img/exposure.jpg}
        \end{figure}
    \end{enumerate}
\end{frame}

\subsection{Objectives}
\begin{frame}{Objectives}
    \setbeamercovered{transparent} 
    \begin{enumerate}
        \item<1> Missing satellite AOD caused by cloud and snow (Aim 1)
            \begin{itemize}
                \item \textcolor[rgb]{0.1,0.1,0.6}{Build a gap-filling model combining snow and cloud features to estimate missing AOD data in the areas with extensive snow and cloud covers}
            \end{itemize}
        \item<2> Insufficient ground PM$_{2.5}$ monitoring stations (Aim 2)
            \begin{itemize}
                \item \textcolor[rgb]{0.1,0.1,0.6}{Establish a calibration model to correct the biases of low-cost PM$_{2.5}$ sensors}
                \item \textcolor[rgb]{0.1,0.1,0.6}{Build a prediction model incorporating the calibrated measurements to predict PM$_{2.5}$ exposures with spatial details}
            \end{itemize}
        \item<3> Effect of short-term PM$_{2.5}$ exposure on renal disease (Aim 3)
            \begin{itemize}
                \item \textcolor[rgb]{0.1,0.1,0.6}{Conduct a time-series study analyzing the association between short-term PM$_{2.5}$ exposure and renal disease\footnote{Emergency department (ED) visits for renal disease} based on the improved PM$_{2.5}$ estimations}
            \end{itemize}
    \end{enumerate}
\end{frame}

\begin{frame}{Objectives}
    \begin{figure}
        \centering
        \includegraphics[width=0.7\textwidth]{img/flow.png}
    \end{figure}
\end{frame}

\section{Specific Aims}
\subsection{Aim 1}
\begin{frame}{Aim 1}{AOD gap-filling with snow/cloud interactions}
    \begin{block}{Approach}
        \begin{itemize}
            \item \textbf{AOD gap-filling model} based on machine learning algorithms\footnote{Random Forests or Neural Networks}
                \begin{align*}
                    \mathrm{AOD_{\mathit{st}}} = & \mathrm{f(Snow/Cloud\;Frac_{\mathit{st}}, Meteorological\;Term_{\mathit{st}},}\\
                    &\mathrm{Landuse\;Term_\mathit{s}, Spatial\;Coord_\mathit{s})}
                \end{align*}
        \end{itemize}
    \end{block}
    \vspace{-15pt}
    \pause
    \begin{block}{Study Domain}
        \begin{itemize}
            \item \textbf{New York State} where there are extensive snow cover and significant missing AOD data
        \end{itemize}
    \end{block}
\end{frame}

\begin{frame}{Aim 1}{AOD gap-filling with snow/cloud interactions}
    \begin{figure}
        \centering
        \includegraphics[height=0.6\textwidth]{img/ny.jpg}
        \label{fig:aim1}
    \end{figure}
\end{frame}

\subsection{Aim 2}
\begin{frame}{Aim 2}{Low-cost sensor calibration and PM$_{2.5}$ exposure assessment}
    \begin{block}{Approach}
        \begin{itemize}
            \item Low-cost sensor calibration
                \begin{itemize}
                    \item \textbf{Polynomial regression model} with measurements from EPA stations and nearby PurpleAir sensors
                    \item \textbf{Kriging with external drift} for the interpolation of calibration coefficients*
                \end{itemize}
            \pause
            \item PM$_{2.5}$ exposure assessment
                \begin{itemize}
                    \item \textbf{Statistical or machine learning models} with calibrated PM$_{2.5}$ measurements
                \end{itemize}
        \end{itemize}
    \end{block}
    \pause
    \begin{block}{Study Domain}
        \begin{itemize}
            \item \textbf{Southern California} where there are dense PurpleAir networks and high PM$_{2.5}$ pollution levels
        \end{itemize}
        
    \end{block}
\end{frame}

\begin{frame}{Aim 2}{Low-cost sensor calibration and PM$_{2.5}$ exposure assessment}
    \begin{figure}
        \centering
        \includegraphics[height=0.55\textwidth]{img/socal.jpg}
        \label{fig:aim2}
    \end{figure}
\end{frame}

\subsection{Aim 3}
\begin{frame}{Aim 3}{Short-term PM$_{2.5}$ exposure and ED visits for renal disease}
    \setbeamercovered{transparent} 
    \begin{block}{Approach}
        \begin{itemize}
        \setlength{\belowdisplayskip}{-10pt}
            \item<1> \textbf{Poisson regression} with distributed lag and ZIP code-specific effects
            \item<2> Two-level Bayesian random-effects meta-analysis
                \begin{enumerate}
                \setlength{\belowdisplayskip}{-10pt}
                    \item First level: \textbf{Poisson regression} for each ZIP code area
                    \setlength{\belowdisplayskip}{-10pt}
                    \item Second level: \textbf{Bayesian meta-analysis} for all ZIP code effects
                \end{enumerate}
        \end{itemize}
    \end{block}
    \begin{block}{Study Domain}
        \begin{itemize}
        \item Extended from Southern California to \textbf{more major cities in California} with high populations and dense PurpleAir networks (\textit{e.g.,} San Francisco Bay Area, Sacramento, and Fresno).
        \end{itemize}
    \end{block}
\end{frame}

\begin{frame}{Aim 3}{Short-term PM$_{2.5}$ exposure and ED visits for renal disease}
    \begin{figure}
        \centering
        \includegraphics[height=0.6\textwidth]{img/ca.jpg}
        \label{fig:aim3}
    \end{figure}
\end{frame}

\section{Q\&A}
\begin{frame}
    \begin{center}
        \LARGE Q \& A
    \end{center}
\end{frame}

% -------------------------------------------- %
% ----------------- Appendix ----------------- %
% -------------------------------------------- %

% All of the following is optional and typically not needed. 
\appendix
\section<presentation>*{\appendixname}

\begin{frame}{Challenges and Possible Solutions}
    \begin{enumerate}
        \item Effectively calibrate and utilize the low-cost sensors in PM$_{2.5}$ prediction
            \begin{itemize}
                \item \textcolor[rgb]{0.1,0.1,0.6}{incorporating related meteorological variables in the calibration model (\textit{e.g.,} temperature and relative humidity)}
                \item \textcolor[rgb]{0.1,0.1,0.6}{incorporating CTM-simulated PM$_{2.5}$ components in order to reflect component-related quality degradation}
                \item \textcolor[rgb]{0.1,0.1,0.6}{Optimal Interpolation (OI), Ensemble Kalman Filtering (EnKF), and the Fourier transform}
            \end{itemize}
        \item the lack of statistical power in the time-series model due to the limited sample size
            \begin{itemize}
                \item \textcolor[rgb]{0.1,0.1,0.6}{extend the study domain to more US states in which the low-cost sensor measurements are available}
                \item \textcolor[rgb]{0.1,0.1,0.6}{combining some small ZIP code areas with similar spatiotemporal patterns of PM$_{2.5}$ pollution (classification)}
            \end{itemize}
    \end{enumerate}
\end{frame}

% -------------------------------------------- %

\subsection*{Aim 1}

\begin{frame}{}
    \begin{table}
        \LARGE
        \centering
        \begin{tabular}{c}
             \textcolor[rgb]{0.1,0.1,0.6}{Aim 1}
        \end{tabular}
    \end{table}
\end{frame}

\begin{frame}{PM$_{2.5}$ Prediction Model}
    \begin{align*}
        \mathrm{PM_{2.5\mathit{st}}=f(Satellite\;AOD_{\mathit{st}},PM_{2.5}\;cov_{\mathit{st}},Meteorological\;term_{\mathit{st}}, } \\
        \mathrm{Landuse\;term_\mathit{s},Month_\mathit{t}, Day_\mathit{t})}
    \end{align*}
\end{frame}

\begin{frame}{Relationship between PM$_{2.5}$ and AOD}
    AOD is defined as the integral of aerosol extinction coefficients along the entire vertical atmospheric column. Its relationship with ground-level PM2.5, assuming a well-mixed boundary layer and no pollution layers aloft, can be expressed as (Koelemeijer et al., 2006)
    \begin{equation*}
        PM_{2.5}=\frac{4\rho r_{eff}}{3f(RH)\times Q_{ext,dry}}\times\frac{f_{PBL}}{H_{PBL}}\times AOD
    \end{equation*}
    where $\rho$ is particle density ($g/m^3$); $r_{eff}$ is particle effective radius; $f(RH)$ is the ratio of ambient and dry extinction coefficients; $Q_{ext,dry}$ is particle extinction efficiency; $f_{PBL}$ is the AOD fraction in the boundary layer; $H_{PBL}$ is the boundary layer height.
\end{frame}

\begin{frame}{Missing AOD in New York State in 2015}
    \begin{table}
        \centering
        \begin{tabular}{c|c|c|c}
            \hline
            Missing Type & Mean & Median & 25$^{th}$ - 75$^{th}$ Quantiles \\
            \hline
            Overall & 90.27\% & 96.54\% & 87.62\% -- 99.84\% \\
            Cloud & 75.58\% & 79.47\% & 64.74\% -- 91.48\% \\
            Snow & 6.14\% & 0\% & 0\% -- 2.48\% \\
            Snow season* & 21.15\% & 14.03\% & 5.23\% -- 31.18\% \\
            \hline
        \end{tabular}
    \end{table}
    {\footnotesize * First 15 weeks of 2015}
\end{frame}

\begin{frame}{Original and Gap-filled AOD in NYS}
    \begin{figure}
        \centering
        \includegraphics[width=0.5\textwidth]{img/appendix/Aim1/figure21.png}
        \includegraphics[width=0.5\textwidth]{img/appendix/Aim1/figure22.png} \\
        \includegraphics[width=0.5\textwidth]{img/appendix/Aim1/figure25.png}
    \end{figure}
\end{frame}

\begin{frame}{Importance of Snow Parameter}
    \begin{figure}
        \centering
        \includegraphics[width=0.5\textwidth]{img/appendix/Aim1/snow_before.png}
        \includegraphics[width=0.5\textwidth]{img/appendix/Aim1/snow_after.png}
        \caption{Variable importance of snow fraction in 1) snow season (left) and 2) the rest of the year (right)}
    \end{figure}
\end{frame}

% -------------------------------------------- %

\subsection*{Aim 2}

\begin{frame}{}
    \begin{table}
        \LARGE
        \centering
        \begin{tabular}{c}
             \textcolor[rgb]{0.1,0.1,0.6}{Aim 2}
        \end{tabular}
    \end{table}
\end{frame}

\begin{frame}{Benchmark Strategy for Low-cost PM$_{2.5}$ Measurements}
    \begin{block}{Step 1}
            \begin{equation*}
        \mathrm{\sqrt{PredPM_\mathit{st}}=\beta_0+\beta_1 \sqrt{MPM_\mathit{t}}+s(X,Y)_\mathit{s}+\epsilon_\mathit{st}}
    \end{equation*}
    \end{block}
    \begin{block}{Step 2}
        \begin{align*}
        \mathrm{PM_{2.5\mathit{st}}=f(Satellite\;AOD_{\mathit{st}},Sensor\;PM_{2.5\mathit{st}},Meteorological\;term_{\mathit{st}},}\\
        \mathrm{Landuse\;term_\mathit{s},Month_\mathit{t}, Day_\mathit{t})}
        \end{align*}
    \end{block}
\end{frame}

\begin{frame}{Polynomial Regression}
    \begin{align*}
        \begin{split}
        y_1=\beta_{n1}x^n+\beta_{(n-1)1}x^{n-1}+&\cdots+\beta_{21}x^2+\beta_{11}x+\beta_{01} + \epsilon_1 \\
        y_2=\beta_{n2}x^n+\beta_{(n-1)2}x^{n-1}+&\cdots+\beta_{22}x^2+\beta_{12}x+\beta_{02} + \epsilon_2 \\
        &\vdots \\
        y_m=\beta_{nm}x^n+\beta_{(n-1)m}x^{n-1}+&\cdots+\beta_{2m}x^2+\beta_{1m}x+\beta_{0m} + \epsilon_m
        \end{split}
    \end{align*}
    \textcolor[rgb]{0.1,0.1,0.6}{$x$ are low-cost sensor measurements; $y_m$ are the measurements from the nearest EPA station. One low-cost sensor corresponds to one equation.}
\end{frame}

\begin{frame}{Errors of Low-cost Sensors}
    \begin{figure}
        \centering
        \includegraphics[width=0.75\textwidth]{img/appendix/Aim2/kelly2017.png}
        \caption{Low-cost sensor PM$_{2.5}$ (PMS) and FEM PM$_{2.5}$ (TEOM). (Kelly et al., 2017)}
    \end{figure}
\end{frame}

\begin{frame}{Errors of Low-cost Sensors}
    \begin{figure}
        \centering
        \includegraphics[width=0.48\textwidth]{img/appendix/Aim2/scatter_reg.jpg}
        \includegraphics[width=0.48\textwidth]{img/appendix/Aim2/hist.png}
        \caption{Left: AQS/Dylos before adjusting; Right: Histogram of errors}
    \end{figure}
\end{frame}

\begin{frame}{Errors of Low-cost Sensors}
    \begin{figure}
        \centering
        \includegraphics[width=\textwidth]{img/appendix/Aim2/time_series.jpg}
        \caption{Time-series for AQS/Dylos PM2.5; \textcolor[rgb]{1,0,0}{Red} -- AQS, \textcolor[rgb]{0,0,1}{Blue} -- Dylos}
    \end{figure}
\end{frame}

\begin{frame}{Collocation}
    \begin{figure}
        \centering
        \includegraphics[width=0.32\textwidth]{img/appendix/Aim2/collocation.png}
    \end{figure}
    We will experiment several distances from near to far to fit the model and then use cross-validation to determine which distance is suitable for the ``collocation''.
\end{frame}

\begin{frame}{Kriging with External Drift}
    \begin{align*}
        \begin{split}
        \hat{\beta}^{ked}_0(s,t)=\sum_{i=1}^m&\lambda^{ked}_{(s_i,t)}\times\beta_0(s_i,t) \\
        \hat{\beta}^{ked}_1(s,t)=\sum_{i=1}^m&\lambda^{ked}_{(s_i,t)}\times\beta_1(s_i,t) \\
        &\vdots \\
        \hat{\beta}^{ked}_n(s,t)=\sum_{i=1}^m&\lambda^{ked}_{(s_i,t)}\times\beta_n(s_i,t) 
        \end{split}
    \end{align*}
    \textcolor[rgb]{0.1,0.1,0.6}{\small The square root of elevation will be used as the external drift because it is a good indicator for PM$_{2.5}$ prediction (Tunno et al., 2016). Other interpolation methods will also be examined, and using cross-validation to determine which method is the most suitable one.}
\end{frame}

\begin{frame}{Interpolations}
    \begin{minipage}{0.58\textwidth}
        \begin{figure}
            \centering
            \includegraphics[width=\textwidth]{img/appendix/Aim2/interpolation.jpg}
        \end{figure}
    \end{minipage}
    \begin{minipage}{0.38\textwidth}
        \small
        \begin{itemize}
            \item Different methods can produce quite different spatial representations and in-depth knowledge of the phenomenon is needed to evaluate which one is the closest to reality.
            \item the preservation of geometrical properties is in some cases more important than actual accuracy.
        \end{itemize}
    \end{minipage}
\end{frame}

\begin{frame}{EPA Stations}
    \begin{figure}
        \centering
        \includegraphics[width=\textwidth]{img/appendix/Aim2/Stations/epa.jpg}
    \end{figure}
\end{frame}

\begin{frame}{PurpleAir Sensors}
    \begin{figure}
        \centering
        \includegraphics[width=\textwidth]{img/appendix/Aim2/Stations/purpleair.jpg}
    \end{figure}
\end{frame}

\begin{frame}{PM$_{2.5}$ Derived from EPA Stations and Low-cost Sensors}
    \begin{figure}
        \centering
        \includegraphics[width=0.7\textwidth]{img/appendix/Aim2/im_location.png}
        \caption{Locations of AQS and Dylos sensors}
    \end{figure}
\end{frame}

\begin{frame}{PM$_{2.5}$ Derived from EPA Stations and Low-cost Sensors}
    \begin{figure}
        \centering
        \includegraphics[width=0.52\textwidth]{img/appendix/Aim2/aqs.jpg}
        \includegraphics[width=0.52\textwidth]{img/appendix/Aim2/aqs_dylos.jpg}
        \caption{Left: PM$_{2.5}$ derived from EPA measurements (6 stations, CV R$^2 = 0.5$); Right: PM$_{2.5}$ derived from EPA and low-cost sensor measurements (45 stations, CV R$^2 = 0.7$)}
    \end{figure}
\end{frame}

\begin{frame}{Mechanism of PurpleAir Sensors}
    \textbf{Optical light scattering/Laser counter}: PurpleAir sensors use a fan to draw air past a laser, causing reflections from any particles in the air. These reflections are used to count particles in six sizes between 0.3 $\mu m$ and 10 $\mu m$ diameter. These particle counts are processed by the sensor to calculate the PM$_1$, PM$_{2.5}$ and PM$_{10}$ mass in $\mu g/m^3$.
    \begin{figure}
        \centering
        \includegraphics[width=0.4\textwidth]{img/appendix/Aim2/sensor.png}
    \end{figure}
\end{frame}

\begin{frame}{High/Low Volume Samplers (FRM)}
    \begin{minipage}[t]{0.49\textwidth}
        \begin{figure}
        \centering
        \includegraphics[width=0.7\textwidth]{img/appendix/Aim2/hivol-sampler.jpg}
    \end{figure}
    \end{minipage}
    \begin{minipage}[t]{0.49\textwidth}
        \begin{enumerate}
            \item The inlet removes particles larger than $10\;\mu m$/$2.5\;\mu m$ by using their greater inertia
            \item Measuring the volume of air sampled and weighing the filters before and after sampling determines the concentration of PM$_{10}$/PM$_{2.5}$ particles in the air.
        \end{enumerate}
    \end{minipage}
\end{frame}

\begin{frame}{Beta Attenuation Monitoring (FEM)}
    \begin{itemize}
        \item BAM employs the absorption of beta radiation\footnote{Beta radiation is a high-energy, high-speed electron or positron emitted by the radioactive decay of an atomic nucleus during the process of beta decay.} by solid particles extracted from air flow. 
        \item The main principle is based on a kind of Bouguer (Lambert–Beer) law: the amount by which the flow of beta radiation (electrons) is attenuated by a solid matter is \textbf{exponentially dependent on its mass} and not on any other feature (such as density, chemical composition or some optical or electrical properties) of this matter.
    \end{itemize}
    
   
\end{frame}

% -------------------------------------------- %

\subsection*{Aim 3}

\begin{frame}{}
    \begin{table}
        \LARGE
        \centering
        \begin{tabular}{c}
             \textcolor[rgb]{0.1,0.1,0.6}{Aim 3}
        \end{tabular}
    \end{table}
\end{frame}

\begin{frame}{Poisson Regression with Distributed Lag}
    \begin{itemize}
        \item Poisson regression with distributed lag and ZIP code-specific effects
            \begin{equation*}
                \mathrm{log(E(Y_{kt}))}=\alpha+\sum_i\beta_i\mathrm{Pollution}_{kti}+\sum_k\lambda_k\mathrm{ZIP}_k+[\mathrm{covariates}]
            \end{equation*}
        \textcolor[rgb]{0.1,0.1,0.6}{The geographical area (ZIP) from which ED counts are spatially aggregated is represented by indicator variables, to control for spatially varying factors and enable the analysis to rely solely on temporal contrasts; this also stringently controls for spatial autocorrelation in the baseline ED visits across the ZIP codes (Sarnat et al., 2013).}
    \end{itemize}
\end{frame}

\begin{frame}{Covariates and Distributed Lags}
    \begin{block}{Covariates}
        \begin{itemize}
            \item Dummy variables for seasons, day of week, and holidays
            \item Dummy variables for hospitals
            \item Polynomial splines for long-term trends and seasonality
            \item Polynomial terms for moving averages of maximum and dew-point temperatures
        \end{itemize}
    \end{block}
    \begin{block}{Distributed lags}
        Lags 0 -- 7 as the \textit{a priori} lag structure since the effect of PM$_{2.5}$ pollution may last a longer period (Sarnat et al., 2015; Ye et al., 2017)
    \end{block}
\end{frame}

\begin{frame}{Sensitivity Analyses}
    \begin{enumerate}
        \item The effect of PM$_{2.5}$ exposure on lag $-1$ \textcolor[rgb]{0.1,0.1,0.6}{(misspecification and potential confounding by temporal factors)}
        \item Different lag structures for exposure \textcolor[rgb]{0.1,0.1,0.6}{(single day lags, moving average of lags, distributed lags 0 -- 3, \textit{etc.})}
        \item Different lag structures for temperatures
        \item Alternating time trend control \textcolor[rgb]{0.1,0.1,0.6}{(different knots specification)}
        \item Co-pollutants using two-pollutant models \textcolor[rgb]{0.1,0.1,0.6}{(potential for confounding of selected single-pollutant results)}
    \end{enumerate}
\end{frame}


\begin{frame}{Bayesian Meta-Analysis}
    \begin{itemize}
       \item Two-level Bayesian random-effects meta-analysis
        \begin{enumerate}
            \item First level: for each ZIP code area$$ \mathrm{log(E(Y_{kt}))}=\alpha+\sum_i\beta_i\mathrm{Pollution}_{kti}+[\mathrm{covariates}]$$
            \item Second level: for all ZIP code effects
            $$\beta_k=\theta_k+\epsilon_k; \quad \epsilon_k\sim N(0, \hat{V}_k)$$
            $$\theta_k\sim N(\alpha_0+\sum_j\gamma_jX_{kj}, \tau^2)$$
            $$\alpha_0+\sum_j\gamma_jX_{kj} \sim N(0, \nu^2)$$
            $$\tau^2\sim \operatorname{Inv-gamma}(a_0, b_0)$$
        \end{enumerate}
    \end{itemize}
\end{frame}

\begin{frame}{PM$_{2.5}$ Composition Estimation}
    \begin{equation*}
        \mathrm{Improved\;component_\mathit{i}=Improved\;PM_{2.5}\times\frac{CTM\;component_\mathit{i}}{CTM\;PM_{2.5}}}
    \end{equation*}
    \begin{itemize}
        \item Using the original composition derived from CTM
        \item Fusing CTM and ground-based composition (Friberg et al., 2016)
    \end{itemize}
\end{frame}

\begin{frame}{Considerations about PM$_{2.5}$ Component Selection}
    \begin{itemize}
        \item We selected species that represented different chemical component classes, which may plausibly confer different toxicities based on different chemical properties
        \item Consideration was also given to species associated with health outcomes in previous studies
        \item The concentrations of the species should be relatively high, with a small samples below the detection limit (BDL) (BDL generally $<5\%$)
    \end{itemize}
\end{frame}

\begin{frame}{Sample Size and Power}
    For a simple Poisson regression (take PM$_{2.5}$ as an example):
    \begin{itemize}
        \item Mean concentration: $15\;\mu g/m^3$
        \item Standard deviation: $7\;\mu g/m^3$
        \item Interquartile range: $9\;\mu g/m^3$
        \item Expected rate ratio: 1.035
        \item $\alpha$-level: 5\% (one-sided, $RR \geq 1.035$)
        \item Power: 0.8
    \end{itemize}
    The minimum sample size is \textbf{8741 counts}. If the average daily counts for ARF is 10 $person/day$ in a city, there should be 874 days' data ($\sim 2.4\;years$).
\end{frame}

\begin{frame}{Kidney Disease Burden Caused by PM$_{2.5}$ Exposure}
    Population attributable fraction (PAF) represents the proportional reduction in population disease that would occur if exposure to PM$_{2.5}$ was reduced to the Environmental Protection Agency's (EPA) recommended levels of 12 $\mu g/m^3$ (Bowe et al., 2018)
\end{frame}

\begin{frame}{Summary Statistics in Atlanta}
    \begin{figure}[H]
        \centering
        \includegraphics[width=\textwidth]{img/appendix/Aim3/table.jpg}
    \end{figure}
\end{frame}

\begin{frame}{PM$_{2.5}$ and Renal Disease in Atlanta (RENAL)}
    \begin{table}
        \small
        \centering
        \begin{tabular}{c|c|c|c}
            \hline
            Pollutant & IQR & RENAL & RENAL\_ANY \\
            \hline
            PM$_{2.5}$ & 8.99 $\mu g/m^3$ & 1.008 (0.987, 1.030) & 1.010 (0.996, 1.024) \\
            EC & 0.68 $\mu g/m^3$ & 1.016 (0.993, 1.039) & 1.014 (1.000, 1.029)** \\
            OC & 1.75 $\mu g/m^3$ & 1.010 (0.989, 1.033) & 1.019 (1.006, 1.033)** \\
            Sulfate & 3.52 $\mu g/m^3$ & 1.001 (0.981, 1.021) & 1.000 (0.987, 1.013) \\
            Nitrate & 0.60 $\mu g/m^3$ & 1.022 (0.994, 1.050) & 1.007 (0.990, 1.024) \\
            \hline
        \end{tabular}
    \end{table}
    {\footnotesize * $0.05 \leq p < 0.10$ \\ ** $p < 0.05$}
\end{frame}

\begin{frame}{PM$_{2.5}$ and Renal Disease in Atlanta (ARF)}
    \begin{table}
        \small
        \centering
        \begin{tabular}{c|c|c|c}
            \hline
            Pollutant & IQR & ARF & ARF\_ANY \\
            \hline
            PM$_{2.5}$ & 8.99 $\mu g/m^3$ & 1.019 (0.962, 1.080) & 1.035 (1.007, 1.063)** \\
            EC & 0.68 $\mu g/m^3$ & 1.019 (0.959, 1.082) & 1.032 (1.004, 1.061)** \\
            OC & 1.75 $\mu g/m^3$ & 1.001 (0.947, 1.059) & 1.035 (1.009, 1.063)** \\
            Sulfate & 3.52 $\mu g/m^3$ & 1.032 (0.977, 1.091) & 1.025 (0.999, 1.052)* \\
            Nitrate & 0.60 $\mu g/m^3$ & 1.051 (0.977, 1.129) & 1.039 (1.006, 1.074)** \\
            \hline
        \end{tabular}
    \end{table}
    {\footnotesize * $0.05 \leq p < 0.10$ \\ ** $p < 0.05$}
\end{frame}

\begin{frame}{Possible mechanisms of PM$_{2.5}$ on kidney disease}
    \begin{enumerate}
        \item Inhaled particles provoke pulmonary inflammation which may then lead to systemic inflammation.
        \item The mechanism involves pollutant-induced disturbances in the lung autonomic nervous system.
        \item Air-borne particulates enter the bloodstream where they may then interact with tissue components to promote the observed pathologic effects. 
    \end{enumerate}
\end{frame}

\begin{frame}{Renal Disease and Other Diseases}
    \begin{itemize}
        \item \textbf{Pulmonary disease}: Several studies have reported consistent association between ARF and dysfunction of extrarenal organs, particularly the lungs.
        \item \textbf{Cardiovascular disease}: Impaired renal function, as determined from the estimated glomerular filtration rate (eGFR), is associated with cardiovascular events and mortality.
        \item \textbf{Diabetes/hypertension}: People with pre-existing chronic diseases such as diabetes or hypertension are more vulnerable to the adverse effect of air pollution.
    \end{itemize}
\end{frame}

\begin{frame}{Primary Diagnoses (RENAL)}
    \begin{figure}
        \centering
        \includegraphics[width=\textwidth]{img/appendix/Aim3/renal_pri.png}
    \end{figure}
\end{frame}

\begin{frame}{Primary Diagnoses (ARF)}
    \begin{figure}
        \centering
        \includegraphics[width=\textwidth]{img/appendix/Aim3/arf_pri.png}
    \end{figure}
\end{frame}

\begin{frame}{Sources of PM$_{2.5}$ Composition}
    \begin{table}
    \centering
    \begin{tabular}{|c|p{0.6\textwidth}|}
        \hline
        \textbf{PM2.5 Composition} & \textbf{Sources}  \\
        \hline
        BC/EC & Combustion sources (primary) \\
        \hline
        VOC & Biogenic emissions, fossil fuel combustion \\
        \hline
        Sulfate &  Sulfur dioxide emissions from power plants, industrial facilities, and traffic sources (secondary) \\
        \hline
        Nitrate & Nitrogen oxides released from power plants, mobile sources, and other combustion sources (secondary) \\
        \hline
        Ammonium & Agricultural sources \\
        \hline 
        Metals &  Oil combustion, crustal dust \\
        \hline
    \end{tabular}
\end{table}
\end{frame}

\end{document}


